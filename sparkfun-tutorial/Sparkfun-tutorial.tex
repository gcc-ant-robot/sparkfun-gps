\documentclass{article}%
\usepackage{amsmath}%
\usepackage{amsfonts}%
\usepackage{amssymb}%
\usepackage{graphicx}
\usepackage[shortlabels]{enumitem}
\usepackage{hyperref}

% Custom Font
\usepackage{mathptmx}

% for double figures
\usepackage{caption}
\usepackage{subcaption}

% Appendix Formatting
\usepackage{appendix}
\appendixtitleon % Appendix Title On
\appendixtitletocon % Appendix TOC Title On

\newcommand{\MYhref}[3][blue]{\href{#2}{\color{#1}{#3}}}%

% Listings code
\usepackage{listings}
\usepackage{color}

\definecolor{dkgreen}{rgb}{0,0.6,0}
\definecolor{gray}{rgb}{0.5,0.5,0.5}
\definecolor{mauve}{rgb}{0.58,0,0.82}

% enviroment for code
\lstset{frame=tb,
  language=java,
  aboveskip=3mm,
  belowskip=3mm,
  showstringspaces=false,
  columns=flexible,
  basicstyle={\small\ttfamily},
  numbers=none,
  numberstyle=\tiny\color{gray},
  keywordstyle=\color{blue},
  commentstyle=\color{dkgreen},
  stringstyle=\color{mauve},
  breaklines=true,
  breakatwhitespace=true,
  tabsize=3
}

\usepackage[utf8]{inputenc}
\usepackage{xcolor}
\usepackage{newunicodechar}

\newcommand\Warning{%
 \makebox[1.4em][c]{%
 \makebox[0pt][c]{\raisebox{.1em}{\small!}}%
 \makebox[0pt][c]{\color{red}\Large$\bigtriangleup$}}}%

\newunicodechar{⚠}{\Warning}

%-------------------------------------------
\newtheorem{theorem}{Theorem}
\newtheorem{acknowledgement}[theorem]{Acknowledgement}
\newtheorem{algorithm}[theorem]{Algorithm}
\newtheorem{axiom}[theorem]{Axiom}
\newtheorem{case}[theorem]{Case}
\newtheorem{claim}[theorem]{Claim}
\newtheorem{conclusion}[theorem]{Conclusion}
\newtheorem{condition}[theorem]{Condition}
\newtheorem{conjecture}[theorem]{Conjecture}
\newtheorem{corollary}[theorem]{Corollary}
\newtheorem{criterion}[theorem]{Criterion}
\newtheorem{definition}[theorem]{Definition}
\newtheorem{example}[theorem]{Example}
\newtheorem{exercise}[theorem]{Exercise}
\newtheorem{lemma}[theorem]{Lemma}
\newtheorem{notation}[theorem]{Notation}
\newtheorem{problem}[theorem]{Problem}
\newtheorem{proposition}[theorem]{Proposition}
\newtheorem{remark}[theorem]{Remark}
\newtheorem{solution}[theorem]{Solution}
\newtheorem{summary}[theorem]{Summary}
\newenvironment{proof}[1][Proof]{\textbf{#1.} }{\ \rule{0.5em}{0.5em}}
\setlength{\textwidth}{7.0in}
\setlength{\oddsidemargin}{-0.35in}
\setlength{\topmargin}{-0.5in}
\setlength{\textheight}{9.0in}
\setlength{\parindent}{0.3in}


%% THEO: Custom Quote Blocks
%\usepackage{showframe,lipsum}
\usepackage[most]{tcolorbox}
\definecolor{block-gray}{gray}{0.85}
\newtcolorbox{myquote}{colback=block-gray,grow to right by=-10mm,grow to left by=-10mm,
boxrule=0pt,boxsep=0pt,breakable}

\begin{document}
\title{GPS-RTK over LoRa Manual}
\author{Theodore Stangebye, Dr. Timothy Mohr,\\
Anna Valenti, \& Matthew Grauff. \\ \\
Grove City College\\ \\
\texttt{theostangebye@gmail.com}, \\
\texttt{\{mohrta, valentiaj18, grauffma18\}@gcc.edu}}
\maketitle
%\begin{flushright}
%\textbf{Theo Stangebye}
%\end{flushright}

%\begin{center}
%\textbf{} \\
%\end{center}


%%%%%%%%%%%%%%%%%% START

%\begin{center}
%	\includegraphics[scale=.5]{./imgs/img0.png}
%\end{center}

% Center table of contents on first page.
\vspace*{\fill}
\tableofcontents
\vspace*{\fill}

\newpage

\begin{figure}
\centering
\begin{minipage}{.5\textwidth}
  \centering
  \includegraphics[height=3in]{./imgs/glamour.png}
  \captionof{figure}{Base Station Mounted on 3D Printed Chassis.}
  \label{fig:glamour}
\end{minipage}%
\end{figure}

\section{Introduction}

This guide will cover the design and implementation of a (relatively) low-cost, high-precision localization solution using Global Positioning System (GPS) Real-Time-Kinematics (RTK).  Here is an overview of the system:

\begin{itemize}
	\item The system is based on the UBLOX F9P GPS Receiver.  We're using the Sparkfun's breakout board, called the ``Sparkfun GPS-RTK 2".  This GPS receiver is accurate to 1-2 cm when in RTK-Fixed mode and is configured using the \href{https://github.com/sparkfun/SparkFun_Ublox_Arduino_Library}{Sparkfun Ublox Arduino Library}.
	\item The system uses LoRa radios to send RTCM data between the Base Station and the rover.  Although untested, the Sparkfun ProRF breakout board has a theoretical line-of-sight range of 2 kilometers!  This module uses the RFM95 radio, which we interface with through the RadioHead library.
\end{itemize}

\begin{myquote}
	Note: on the two units which I am sending to Grove City, the LCD display is connected to the Blackboard.  This is because I didn't have my soldering iron handy to solder these displays to the ProRF.  The I2C ports on the Blackboard are directly connected to the Qwiic bus, so the Blackbaord is basically a ``passthrough" which physically connects the LCD panels to the ProRF via the Qwiic bus.
\end{myquote}

\subsection{Functional Overview}
This guide does not attempt to explain how RTK works.  However, it is sufficient to say that RTK localization systems consist of a bast station, which typically does not move, and which generates correction data which is then sent to a ``rover" GPS receiver. 
The Rover uses the Base Station's correction data to refine its location solution, typically to around 1-2 cm accuracy for RTK systems.
\subsubsection{Base Station}
\begin{enumerate}
	\item The Base Station is not intended to move.  When the unit is first powered on, the Base Station's GPS receiver is ``reset".  This step is intended to purge any remnants from previous runs, and resets the GPS receiver to a known state.

	\item Next, the Base Station begins a ``Survey-in" routine.  As currently configured, the Base Station will survey its location for at least 1 minute, and will continue surveying until it the standard deviation of its own position solutions is less than 2.5 meters.
	\item Once the survey-in step is complete, the Base Station will go into TIME mode.  In this mode, the Base Station will use its survey'ed location as a ground truth and will generate RTCM (GPS correction) messages for other GPS receivers.  The messages generated by our Base Station are tailored to our specific application and are shown in Table \ref{tab:enabled-rtcm}\footnote{I spent some time experimenting to come up with a few messages that would get RTK-Fixes while minimizing the load on the LoRa radios.  Please see \href{https://www.ardusimple.com/configuration-files/}{Ardusimple's Configuration Files} for instructions on delimiting RTCM messages to conserve bandwidth.  Also look at \href{https://www.u-blox.com/sites/default/files/ZED-F9P_IntegrationManual_\%28UBX-18010802\%29.pdf}{Section 3.1.5 of the Ublox F9P integration manual} for information on what each message is used for.}
\end{enumerate}

\begin{table}[]
\centering
\begin{tabular}{|cll|}
\hline
\multicolumn{1}{|c|}{\textbf{RTCM Message Label}} & \multicolumn{1}{l|}{\textbf{Description}} & \textbf{Frequency} \\ \hline
1005 & Stationary RTK reference station ARP & every 10 seconds \\
1074 & GPS MSM4 & every second \\
1084 & GLONASS MSM4 & every second \\
1230 & GLONASS code-phase & every 5 seconds \\ \hline
\end{tabular}
\caption{These RTCM messages are generated by the Base Station.}
\label{tab:enabled-rtcm}
\end{table}

\subsubsection{Rover}
When first powered on, the F9P unit is ``hard reset" by the ProRF in an attempt to reset the unit to a known state. Then, the GPS receiver immediately begins searching for a location fix.  The rover will show the fix type, the RTK status: (None, Float, or Fixed\footnote{RTK=NONE indicates that no RTK solution is being used.  This may be the result of a loss of radio link, or a failure on the part of the Base Station to provide correction data.  RTK=Float means that correction data is being used, but there is still some uncertainty in the solution.  RTK=Fixed is the gold standard, and usually means that your solution is accurate to the cm level.  See the \href{https://www.u-blox.com/sites/default/files/ZED-F9P_IntegrationManual_\%28UBX-18010802\%29.pdf}{Ublox F9P integration manual} for a much more detailed explanation.}), as well as the positional accuracy and distance from Base Station.

\section{System Description}
\subsection{Hardware Overview}
The hardware of this system is comprised of two (almost) identical units: the Base Station (on black 3d printed chassis) and the Rover (on a white chassis).  Block diagrams for both units are in the appendices.  \textbf{The only physical difference} between the units is is the ProRF-to-F9P wiring:
	\begin{itemize}
		\item The Base Station has a UART wire going from the F9P board's \texttt{UART1} TX pin, to the ProRF's \texttt{Serial1} RX pin.
		\item The Rover has a UART wire going from the ProRF's \texttt{Serial1} TX pin, to the F9P's \texttt{UART2} RX pin. 
	\end{itemize}
	This difference aside, the two units are identical.  Both units: 
	\begin{itemize}
		\item Have a ``Qwiic" connection between the ProRF and the F9P.  This connection is used by the ProRF to configure the F9P at startup.  This connection is \textbf{not} used to relay RTCM data from the F9P to the ProRF.
		\item Both have UFL to SMA connectors for the ProRF and the F9P. (Note, that the F9P uses a different UFL to SMA connector since the SMA pin mating is different). 
		\item Both units are connected to a Sparkfun Blackboard via the Qwiic bus. \textbf{The Blackboard is NOT USED!}.  There is no source code for the Blackbaord.  On these units, it exists only to provide power from the Blackboard's DC barrel jack to the other components via the Qwiic bus.
	\end{itemize}
\subsection{Software Overview}
There are only two source files for this project: \texttt{Theo\_Base.ino} and \texttt{Theo\_Rover.ino}, for the Base Station ProRF and Rover ProRF microcontrollers respectively. 
I've tried to make the system robust so that all of the system configuration, setup, and runtime happenings are contained in these source code files.  This should make the system easier to comprehend (only one source code file per unit) and simpler to operate (flash and go!).  \textbf{Basically, everything you need software-wise is in these two .C files!}

\begin{myquote} See the \textbf{PROJECT REPOSITORY} for all the files and resources used in this project:\\ \textbf{\url{https://github.com/gcc-ant-robot/sparkfun-gps}}
\end{myquote}

\subsubsection{Base Station Software}

\texttt{Theo\_Base.ino} is meant to be used on the Base Station Pro-RF module.  On startup, the ProRF will open communication lines with the F9P and its onboard LoRa radio as seen in the configuration section of the \texttt{Theo\_Base.ino} source code file.  It will reset the F9P module (clearing out previous Base Station location solutions, etc.) and will begin the Survey-In mode on the F9P.  

Once Survey-In is complete, the ProRF begins broadcasting RTCM data, by reading bytes of RTCM from the ProRF's \texttt{Serial1} buffer, and shuttling that data to the LoRa radio driver instance: \texttt{rf95}.

One important thing to note is that RTCM data is ``bursty", there's alot of data for a few milliseconds, and then there's a lull for ~950 ms until the next group of RTCM messages are generated by the F9P.  (Remember that Table \ref{tab:enabled-rtcm} shows how often each RTCM message is sent).   In order to operate efficiently (or at all... as I found out), we cannot sent each RTCM byte as its own LoRa message.
Thus, RTCM data is grouped into ``packets" that are 250 Bytes long, and then each packet is sent as a separate LoRa message.

The ProRF's \texttt{Serial1} is setup with a timeout of 100ms, so that when an RTCM byte is received after a period of inactivity (the ~950ms that the F9P is not generating RTCM messages), the \texttt{Serial1} interface will wait 100ms for more bytes from the F9P before \texttt{Serial1.available()} becomes TRUE. Basically, the 100ms timeout let's us say something like: ``When you get a byte from the F9P, wait 100ms for more bytes from the F9P before you start packetizing them to send over LoRa".  This ensures that we send complete (uncorrupted) RTCM messages over the LoRa link.


To summarize the RTCM handling portion of the code:
\begin{enumerate}
	\item The F9P generates RTCM messages roughly once per second.  This data arrives at the ProRF's \texttt{Serial1} port very quickly (basically all at once).
	\item The \texttt{Serial1} interface waits 100ms from the time it first receives an RTCM byte to make sure that all the information has been received.
	\item The RTCM data for that second is packetized into LoRa messages that are 250 bytes in length.
	\item These messages are sent out over the LoRa link during the ~950ms of inactivity which exists before the next batch of RTCM messages are generated by the F9P.
\end{enumerate}   
\begin{myquote}It is generally accepted that you should get RTCM messages from the Base Station to the Rover within 1 second of their creation in order to get an RTK FIXED solution at the rover.
\end{myquote}

\subsubsection{Rover Software}
The \texttt{Theo\_Rover.ino} source code file is intended to be run on the Rover's ProRF module.  Just like the Base Station, the Rover ProRF opens communication lines to the LCD display, F9P GPS receiver, and onboard LoRa radio. The ProRf also resets the F9P module (to get a consistent starting point), and then starts waiting to receive RTCM messages.

The Rover's RTCM handling code shuttles data from the LoRa radio to the ProRF's \texttt{Serial1} interface.  This data is a burst of RTCM messages, which are spread across multiple LoRa messages (or packets).  In order to piece these LoRa messages back into a burst of RTCM data, the code will wait for \texttt{RFWAITTIME} ms (currently set to 500ms) after the first LoRa message is received, before sending the combined messages to the Rover F9P all at once over the ProRf's \texttt{Serial1} port.

To summarize the RTCM handling code for the Rover:
\begin{enumerate}
	\item When a LoRa message is first received, take the message's data and place it in a buffer.
	\item For the next 500 ms, take any LoRa messages which are received, and concatenate these to the end of the buffer from the previous step.  (Basically, recombine the LoRa messages into a burst of RTCM data).
	\item Once the 500 ms is up, take all the data in the buffer and write it to the \texttt{Serial1} interface, which is connected to the F9P.
\end{enumerate}
\begin{myquote}
The Rover source code also has a \texttt{check\_gps()} method which demonstrates how to query the F9P for its solution and status.  This method is used to update the Rover's LCD display.
\end{myquote}

\section{Installation Instructions}
\begin{myquote}
⚠ WARNING: Never power up Base Station or Rover without their LoRa and GPS antennas! ⚠
\end{myquote}


\subsection{Prerequisites}
You'll want to make sure that you have the following installed:
\begin{itemize}
	\item Arduino IDE: \url{https://www.arduino.cc/en/main/software}
	\item Ublox U-Center: \url{https://www.u-blox.com/en/product/u-center}
	\item Latest F9P Firmware: See the ``Firmware Update" section here: \\ \url{https://www.u-blox.com/en/product/zed-f9p-module#tab-documentation-resources} \\ At the time of writing, the latest firmware is ``\href{https://www.u-blox.com/en/ubx-viewer/view/UBX\_F9\_100\_HPG\_113\_ZED\_F9P.7e6e899c5597acddf2f5f2f70fdf5fbe.bin?url=https\%3A\%2F\%2Fwww.u-blox.com\%2Fsites\%2Fdefault\%2Ffiles\%2FUBX\_F9\_100\_HPG\_113\_ZED\_F9P.7e6e899c5597acddf2f5f2f70fdf5fbe.bin}{ZED-F9P HPG 1.13}". 
	\item ProRF Board Support: Follow the tutorial in the ``Setting Up Arduino" section here: \\ \url{https://learn.sparkfun.com/tutorials/sparkfun-samd21-pro-rf-hookup-guide?_ga=2.144736500.1907947329.1598414611-1867689729.1598414611#setting-up-arduino}
	\item You'll need to install the following Arduino Libraries:
	\begin{myquote}
	See here for instructions on installing Arduino Libraries: \\ \url{https://www.arduino.cc/en/Guide/Libraries#how-to-install-a-library}	
	\end{myquote}

		\item RadioHead Arduino Library: This webpage has a link to the library in ZIP form: \\ \url{https://www.airspayce.com/mikem/arduino/RadioHead/}
		\item Ublox Arduino Library: I used Version 1.8.6 here: \\ \url{https://github.com/sparkfun/SparkFun_Ublox_Arduino_Library/releases/tag/v1.8.6}
\end{itemize}

\subsection{Base Station Setup}
I'll now explain how to repeat the Base Station setup/installation for yourself.
\subsubsection{Upgrade Firmware on Base Station F9P}
\begin{enumerate}
	\item Begin with a Base Station that is physically assembled:
	\begin{itemize}
		\item The ProRF and F9P are connected via Qwiic and Serial lines.
		\item The LoRa and GPS antennas are attached.
		\item The LCD display attached to the I2C bus if desired.
		\item Blackboard, or other power-source, connected to the Qwiic bus for power distribution.
	\end{itemize}
	\item Connect the F9P to your laptop with a USB-A to USB-C cable.
	\item Open U-Center on your laptop.
	\item Under the Receiver menu, select Connection $\rightarrow$ COM15. (You may see a different COM port, select whichever COM port is available.)
	\item You should begin to see data/text/images update in the U-Center window with live GPS data.
	\item Under the ``Tools" menu, select ``Firmware Update".
	\item In the ``Firmware Image" dialog box, select the firmware binary which you downloaded from Ublox (It should have been the only ``.bin" file you downloaded in the prerequisites section).
	\item Click the green ``Go" sign in the bottom left of the window to start the firmware update.  It may take a minute or two for the update to complete.
	\item Once the update is complete, ``Bounce" the F9P (unplug, and replug it to your laptop and reconnect to U-Center as in step 3).
	\item You can now unplug the F9P from your laptop.
\end{enumerate}
\subsubsection{Program Base Station ProRF}
\begin{enumerate}
	\item Connect the ProRF to your laptop with a micro-USB to USB-A cable.
	\item Open the Arduino IDE.
	\item In the ``Tools" Menu of the Arduino IDE, make sure that ``Sparkfun SAMD21 ProRF" is selected for the ``Board" option.
	\item Also in the ``Tools" Menu, make sure that you select whatever port your ProRF is connected to your laptop on.  For me, this is ``/dev/cu.usbmodem5998".
	\item Open the \texttt{Theo\_Base.ino} source code file, and hit the Upload arrow to flash this to the ProRF.
	\item Once the source code is flashed, you're done!
\end{enumerate}

\subsection{Rover Setup}
The setup for the rover is pretty similar to the Base Station:
\subsubsection{Upgrade Firmware on Rover F9P}
The F9P Firmware update process is identical to that of the Base Station:
\begin{enumerate}
	\item Begin with a Base Station that is physically assembled:
	\begin{itemize}
		\item The ProRF and F9P are connected via Qwiic and Serial lines.
		\item The LoRa and GPS antennas are attached.
		\item The LCD display attached to the I2C bus if desired.
		\item Blackboard, or other power-source, connected to the Qwiic bus for power distribution.
	\end{itemize}
	\item Connect the F9P to your laptop with a USB-A to USB-C cable.
	\item Open U-Center on your laptop.
	\item Under the Receiver menu, select Connection $\rightarrow$ COM15. (You may see a different COM port, select whichever COM port is available.)
	\item You should begin to see data/text/images update in the U-Center window with live GPS data.
	\item Under the ``Tools" menu, select ``Firmware Update".
	\item In the ``Firmware Image" dialog box, select the firmware binary which you downloaded from Ublox (It should have been the only ``.bin" file you downloaded in the prerequisites section).
	\item Click the green ``Go" sign in the bottom left of the window to start the firmware update.  It may take a minute or two for the update to complete.
	\item Once the update is complete, ``Bounce" the F9P (unplug, and replug it to your laptop and reconnect to U-Center as in step 3).
	\item You can now unplug the F9P from your laptop.
\end{enumerate}
\subsubsection{Program Rover ProRF}
This process is also almost identical to that of the Base Station, EXCEPT, we will use the \textbf{\texttt{Theo\_Rover.ino}} source code file.
\begin{enumerate}
	\item Connect the ProRF to your laptop with a micro-USB to USB-A cable.
	\item Open the Arduino IDE.
	\item In the ``Tools" Menu of the Arduino IDE, make sure that ``Sparkfun SAMD21 ProRF" is selected for the ``Board" option.
	\item Also in the ``Tools" Menu, make sure that you select whatever port your ProRF is connected to your laptop on.  For me, this is ``/dev/cu.usbmodem5998".
	\item Open the \texttt{Theo\_Rover.ino} source code file, and hit the Upload arrow to flash this to the ProRF.
	\item Once the source code is flashed, you're done!
\end{enumerate}

\section{Test!}
Go outside with your Base Station and rover.  Connect a 12V battery to the Base Station Blackboard once your've set it in a semi-permanent location.  Wait for it to ``Survey-In". Then, connect your Rover to a battery, and see how far away your can walk before you loose your RTK Fixed solution.

%\appendix
\begin{appendices}
\section{Block Diagram}
See last page.

\begin{figure}[t]
\centering
\begin{minipage}{1.0\textwidth}
  \centering
  \includegraphics[height=9.5in]{./imgs/bdiagram.pdf}
  \captionof{figure}{Block Diagram}
  \label{fig:diagram}
\end{minipage}%
\end{figure}

\section{References}
\begin{itemize}
	\item \textbf{Project Repository:} Contains the source code, 3d printer files, and everything else Theo used in this project: \url{https://github.com/gcc-ant-robot/sparkfun-gps}
	\item Ublox F9P Integration Manual, Section 3.1.5 contains detailed information about what RTCM messages, Fixed vs. Float solution types, and more: \\ \url{https://www.u-blox.com/sites/default/files/ZED-F9P_IntegrationManual_\%28UBX-18010802\%29.pdf}
	\item Ardusimple's F9P Configuration Files, show handy advice for delimiting RTCM data for the F9P: \url{https://www.ardusimple.com/configuration-files/}
	\item Radiohead Documentation for RFM95: \url{https://www.airspayce.com/mikem/arduino/RadioHead/classRH__RF95.html#details}
\end{itemize}

\section{Index of Terms}
\begin{itemize}
	\item ``MSM4": ``MSM4" data contains the code, phase, and carrier-to-noise measurements for the corresponding satellite constellation (GPS, Glonass, Galileo, etc...).
	\item ``RTK": Real-time-kinematics.
	\item F9P: Usually refers to the Sparkfun GPS-RTK board, which uses a Ublox F9P GPS receiver.
	\item ``Glonass": Russian GPS constellation.
	\item ``GPS": Generally, this is the technology which allows us to triangulate our location.  However, technically, this is the US/DOD constellation of satellites which enable this function. 
\end{itemize}
\end{appendices}

\end{document}